% Tamanhos
% \tiny
% \scriptsize
% \footnotesize
% \small 
% \normalsize
% \large 
% \Large 
% \LARGE 
% \huge
% \Huge

% Posicionamento
% \centering 
% \raggedright
% \raggedleft
% \vfill 
% \hfill 
% \vspace{Xcm}   % Colocar * caso esteja no começo de uma página. Ex: \vspace*{...}
% \hspace{Xcm}

% Estilo de página
% \thispagestyle{<<nosso>>}
% \thispagestyle{empty}
% \thispagestyle{plain}  (só número, sem cabeço)
% https://www.overleaf.com/learn/latex/Headers_and_footers

% Compilador que permite usar fonte de sistema: xelatex, lualatex
% Compilador que não permite usar fonte de sistema: latex, pdflatex

% Definindo fontes
% \setmainfont{Times New Roman}  % Todo o texto
% \newfontfamily\avenir{Avenir}  % Contexto

\begingroup\thispagestyle{empty}\vspace*{-.01\textheight}\parindent=0pt 
              \formular
              \huge 
              \textbf{No princípio\\era o meio}\\\baselineskip=.67\baselineskip 

              \medskip
              
              \LARGE
              Luiz Tatit
              
              \vspace{4cm}              

              \newfontfamily\minion{Minion Pro}
              {\selectfont\minion\small Luís Augusto Fischer (\textit{organização})}
              
              \vspace{0.5cm}

              {\selectfont\minion\footnotesize
              1ª edição}
                    
              \vfill
              
              \begin{wrapfigure}{r}{6.5cm}
              \vspace*{-1.22\baselineskip}
              \includegraphics[width=3.5cm]{./logoacorde.png}
              \end{wrapfigure}

              \newfontfamily\timesnewroman{Times New Roman}
              {\fontsize{30}{40}\selectfont \timesnewroman hedra}
              
              \medskip

              {\selectfont\minion\small
              São Paulo \quad\the\year}
\endgroup
\pagebreak

\begingroup 

\footnotesize\parindent0pt\parskip5pt\thispagestyle{empty} 
\vspace*{.1\textheight}\mbox{} \vfill
\baselineskip=.92\baselineskip
\thispagestyle{empty}

\textbf{Luiz Tatit} (1951) é professor de linguística, crítico, ensaísta e teórico da canção, além de cancionista. Depois da atuação no grupo Rumo, um dos experimentos mais vigorosos e relevantes da vanguarda paulista, fez carreira solo com o álbum \textit{Felicidade} (1997), passando por outros como \textit{O meio} (2000), \textit{Ouvidos uni-vos} (2005), \textit{Rodopio} (2007), \textit{Sem destino} (2010), \textit{Palavras e sonhos} (2016), entre outros. Trabalhou também em parcerias, como a produzida junto a Arrigo Barnabé e Lívia Nestróvski no álbum \textit{De nada mais a algo além} (2014), e o conjunto de shows com José Miguel Wisnik e Arthur Nestróvski, no projeto \textit{O fim da canção} (2013). O trabalho como crítico começou a partir dos anos 1980, com textos escritos para cadernos culturais, além da publicação do livro \textit{A canção, eficácia e encanto} (1986). Publicou também \textit{O cancionista} (1996), \textit{Musicando a semiótica} (1997), \textit{O século da canção} (2004) e \textit{Todos entoam} (2014), sempre tendo como objeto a canção e a construção de uma teoria e mediação próprias a essa linguagem artística.

\textbf{No princípio era o meio} reúne textos, divididos em três blocos temáticos, que desenvolvem a singularidade da canção como linguagem artística, tanto no aspecto formal, quanto histórico e técnico. O primeiro bloco define o sentido da conceituação. O segundo é formado por dois estudos de caso de artistas vinculados a diferentes formas de composição, movimentos artísticos e contextos históricos distintos: Tom Jobim e Itamar Assumpção. O terceiro, por sua vez, diferente do tom ensaístico dos dois primeiros blocos, é formado por textos com linguagem mais próxima à do jornalismo cultural.

\textbf{Marcos Lacerda} é sociólogo e ensaísta. Foi diretor de música da Funarte, responsável por políticas de âmbito nacional. Publicou como autor o livro \textit{Hotel Universo: a poética de Ronaldo Bastos} (2019); e como organizador os livros \textit{Música: ensaios brasileiros contemporâneos} (2016) e \textit{A canção como música de invenção} (2018). É um dos curadores da coleção Cadernos Ultramares e um dos editores da revista de crítica musical \textit{Uma canção}. 

\endgroup
\pagebreak



