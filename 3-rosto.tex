% Tamanhos
% \tiny
% \scriptsize
% \footnotesize
% \small 
% \normalsize
% \large 
% \Large 
% \LARGE 
% \huge
% \Huge

% Posicionamento
% \centering 
% \raggedright
% \raggedleft
% \vfill 
% \hfill 
% \vspace{Xcm}   % Colocar * caso esteja no começo de uma página. Ex: \vspace*{...}
% \hspace{Xcm}

% Estilo de página
% \thispagestyle{<<nosso>>}
% \thispagestyle{empty}
% \thispagestyle{plain}  (só número, sem cabeço)
% https://www.overleaf.com/learn/latex/Headers_and_footers

% Compilador que permite usar fonte de sistema: xelatex, lualatex
% Compilador que não permite usar fonte de sistema: latex, pdflatex

% Definindo fontes
% \setmainfont{Times New Roman}  % Todo o texto
% \newfontfamily\avenir{Avenir}  % Contexto

\begingroup\thispagestyle{empty}\vspace*{-.01\textheight}\parindent=0pt 
              \formular
              \huge 
              \textbf{No princípio\\era o meio}\\\baselineskip=.67\baselineskip 

              \medskip
              
              \LARGE
              Luiz Tatit
              
              \vspace{4cm}              

              \newfontfamily\minion{Minion Pro}
              {\selectfont\minion\small Luís Augusto Fischer (\textit{organização})}
              
              \vspace{0.5cm}

              {\selectfont\minion\footnotesize
              1ª edição}
                    
              \vfill
              
              \begin{wrapfigure}{r}{6.5cm}
              \vspace*{-1.22\baselineskip}
              \includegraphics[width=3.5cm]{./logoacorde.png}
              \end{wrapfigure}

              \newfontfamily\timesnewroman{Times New Roman}
              {\fontsize{30}{40}\selectfont \timesnewroman hedra}
              
              \medskip

              {\selectfont\minion\small
              São Paulo \quad\the\year}
\endgroup
\pagebreak

\begingroup 

\footnotesize\parindent0pt\parskip5pt\thispagestyle{empty} 
\vspace*{.1\textheight}\mbox{} \vfill
\baselineskip=.92\baselineskip
\thispagestyle{empty}

\textbf{Luiz Tatit} é professor de linguística, crítico, ensaísta e teórico da canção, além de cancionista. Depois da bem-sucedida atuação no grupo Rumo, um dos experimentos mais vigorosos e relevantes da Vanguarda Paulista, começa a desenvolver sua carreira solo com o álbum \textit{Felicidade}, em 1997, passando por álbuns como \textit{O meio} (2000), \textit{Ouvidos uni-vos} (2005), \textit{Rodopio} (2007), \textit{Sem destino} (2010), \textit{Palavras e sonhos} (2016), entre outros. Tem também uma série de trabalhos em parcerias, como a que fez com Arrigo Barnabé e Lívia Nestróvski no álbum \textit{De nada mais a algo além} (2014), e o conjunto de shows com José Miguel Wisnik e Arthur Nestróvski, no projeto \textit{O fim da canção} (2013). Seu trabalho na crítica começa a se desenvolver a partir dos anos 1980, com a escrita de textos em cadernos culturais, e a publicação do livro \textit{A canção, eficácia e encanto} (1986). Depois se seguiu uma série de outras publicações, entre livros e artigos, cabendo aqui destacar, entre muitos, os seguintes: \textit{O cancionista} (1996), \textit{Musicando a semiótica} (1997), \textit{O século da canção} (2004) e \textit{Todos entoam} (2014), tendo sempre como objeto central a canção, e como objetivo geral a construção de uma teoria e de uma mediação crítica próprias a essa linguagem artística.

\textbf{No princípio era o meio} reúne textos nos quais o desenvolvimento de um movimento de conceituação que tem como objetivo central explicitar a singularidade da canção como linguagem artística, ou, se seguirmos os versos da canção mencionada: \textit{da voz que fala pela fala e pela voz}. E isso tanto no aspecto formal, como histórico e técnico. No desenrolar dos artigos e ensaios selecionados, perceberemos o modo como o nosso autor conduziu a sua reflexão. Para isso, dividimos o livro em três blocos. \textit{O primeiro} define o sentido mesmo da conceituação, apresentando a sua relação com os aspectos formais, históricos e técnicos. Estão inclusos aqui os textos ``O século \textsc{xx} em foco'' (2004) e ``A dicção do cancionista'' (2002). \textit{O segundo}, formado por dois estudos de caso, cujas obras exemplificam e explicitam aspectos da teoria da canção do primeiro bloco. Selecionamos, estrategicamente, dois artistas vinculados a formas de composição, movimentos artísticos e contextos históricos distintos: Tom Jobim e Itamar Assumpção, com os textos respectivos, ``A dicção de Tom Jobim''(2002) e ``A transmutação do artista'' (2007). Em ambos os blocos, o tom da escrita é mais analítico, denotando a necessidade de precisão conceitual e, mesmo, precaução metodológica. \textit{O terceiro}, por sua vez, destaca textos com um teor mais propriamente ensaístico, com temática mais geral, embora tendo a autonomia da canção como regulador e parâmetro central. Ele é formado por ensaios mais soltos que, a sua maneira, condensa as problemáticas apresentadas no primeiro e segundo blocos, só que com uma linguagem mais próxima do jornalismo cultural, a que estão mais acostumados os leitores de crítica da canção, ao menos no Brasil. São desse bloco os textos ``Vocação e perspectiva dos cancionistas'', ``O momento de criação em canção popular'' e ``Cancionistas invisíveis''. Essa variação nos textos é bastante significativa para os nossos propósitos. Ora, o que busca a coleção é contribuir para uma sistematização da produção crítica em canção no país, tendo em vista a sua diversidade temática e analítica, muitas vezes não explicitada. Tatit pode ser considerado o conceituador mais contundente e preciso da canção como linguagem artística autônoma. É, também, precursor desse tipo de estudo e, digamos assim, dessa linhagem da crítica da canção.

\textbf{Luís Augusto Fischer} é professor titular de Literatura Brasileira no Instituto de Letras da \textsc{ufrgs}, onde leciona desde 1984. Em 1992 criou um curso optativo de Canção Popular Brasileira, para alunos de Letras, Ciências Humanas e Música, que funciona desde então. A partir dele, criou-se uma abertura para estudos e pesquisas de pós-graduação, onde tem orientado, com trabalhos realizados já há duas décadas. Articulou a criação do Núcleo de Estudos da Canção, junto à Pró-Reitoria de Extensão da UFRGS, que mantém programação desde então, com palestras, depoimentos e debates entre músicos, cancionistas, estudiosos e interessados. Junto com Guto Leite, organizou o livro O alcance da canção, pela editora Arquipélago (2016), que reuniu uma série de estudos realizados no Instituto em torno do tema. É autor de uma série de ensaios, artigos e resenhas sobre o universo da canção, publicados em jornais, revistas e livros.

%\textbf{Certas canções}

\endgroup
\pagebreak