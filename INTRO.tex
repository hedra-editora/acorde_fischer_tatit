\textbf{Apresentação }

Luiz Tatit é professor de linguística, crítico, ensaísta e teórico da
canção, além de cancionista. Depois da bem-sucedida atuação no grupo
Rumo, um dos experimentos mais vigorosos e relevantes da Vanguarda
Paulista, começa a desenvolver sua carreira solo com o álbum
``Felicidade'', em 1997, passando por álbuns como ``O Meio'' (2000),
``Ouvidos uni-vos'' (2005), ``Rodopio'' (2007), ``Sem destino'' (2010),
Palavras e sonhos (2016), entre outros. Além disso, tem uma série de
trabalhos em parcerias, como os que fez com Arrigo Barnabé e Lívia
Nestróvski (De nada mais a algo além, 2014), e o conjunto de shows com
José Miguel Wisnik e Arthur Nestróvski, no projeto ``O fim da canção''
(2013).

O seu trabalho na crítica começa a se desenvolver a partir dos anos 80,
com a escrita de textos em cadernos culturais, e a publicação do livro
``A canção, eficácia e encanto'' (1986). Depois se seguiu uma série de
outras publicações, entre livros e artigos, cabendo aqui destacar, entre
muitos outros, os seguintes: ``O Cancionista (1996), ``Musicando a
semiótica'' (1997), ``O século da canção'' (2004) e ``Todos Entoam''
(2014), tendo sempre como objeto central a canção, e como objetivo geral
a construção de uma teoria e de uma mediação crítica própria a essa
linguagem artística.

Uma composição sua, em parceria com José Miguel Wisnik, gravada no
primeiro álbum desse autor, que é também crítico, ensaísta, pensador da
cultura, teórica da canção, ``Mestres cantores'', explicita bem a sua
condição como cancionista e intelectual, artista da canção e professor
universitário, entre escalas e escolas, entre o céu do pensamento e o
domínio sublunar do cotidiano

Nós aqui livres docentes

Docemente livres

Entre o rap e o repente

A canção dolente

A canção enquanto tal

A música total

Da voz que fala pela fala

E pela voz

O que nós veremos nos textos selecionados para este livro será,
justamente, o desenvolvimento de um movimento de conceituação que tem
como objetivo central explicitar a singularidade da canção como
linguagem artística ou se seguirmos os versos da canção mencionada:
\emph{da voz que fala pela fala e pela voz}. E isso tanto no aspecto
formal, como histórico e técnico. No desenrolar dos artigos e ensaios
selecionados, perceberemos o modo como o nosso autor conduziu a sua
reflexão. Para isso, dividimos o livro em três blocos. \emph{O primeiro}
define o sentido mesmo da conceituação, apresentando a sua relação com
os aspectos formais, históricos e técnicos. Estão inclusos aqui os
textos "\emph{O século XX em foco}" (2004) e ``\emph{A dicção do
cancionista"} (2002).

\emph{O segundo}, formado por dois estudos de caso, cujas obras
exemplificam e explicitam aspectos da teoria da canção do primeiro
bloco. Selecionamos, estrategicamente, dois artistas vinculados a formas
de composição, movimentos artísticos e contextos históricos distintos:
Tom Jobim e Itamar Assumpção, com os textos respectivos, "\emph{A dicção
de Tom Jobim}"(2002) e "\emph{A transmutação do artista}" (2007).

Em ambos os blocos, o tom da escrita é mais analítico, denotando a
necessidade de precisão conceitual e, mesmo, precaução metodológica.
\emph{O terceiro}, por sua vez, destaca textos com um teor mais
propriamente ensaístico, com temática mais geral, embora tendo a
autonomia da canção como regulador e parâmetro central. Ele é formado
por ensaios mais soltos que, a sua maneira, condensa as problemáticas
apresentadas no primeiro e segundo blocos, só que com uma linguagem mais
próxima do jornalismo cultural, a que estão mais acostumados os leitores
de crítica da canção, ao menos no Brasil. São desse bloco os textos
``\emph{Vocação e perspectiva dos cancionistas''}, ``\emph{O momento de
criação em canção popular}'' e ``\emph{Cancionistas invisíveis''.}

Essa variação nos textos é bastante significativa para os nossos
propósitos. Ora, o que busca a coleção é contribuir para uma
sistematização da produção crítica em canção no país, tendo em vista a
sua diversidade temática e analítica, muitas vezes não explicitada.
Tatit pode ser considerado o conceituador mais contundente e preciso da
canção como linguagem artística autônoma. É, também, precursor desse
tipo de estudo e, digamos assim, dessa linhagem da crítica da canção.

\textbf{A delimitação conceitual}

Comecemos então pelos textos do primeiro bloco, seguindo diretamente a
ordem do livro. Eles sistematizam e, ao mesmo tempo, historicizam a
canção popular moderna, se concentrando no Brasil.

Tudo passa pelo processo que vai se dar entre a fala e o canto. O
movimento que vai gerar algo novo, que não é nem o canto em si, nem a
fala. A travessia entre estas duas instâncias é a que vai criar
propriamente a canção, como linguagem artística autônoma, em relação à
literatura e em relação à música. Não sendo literatura, nem música, a
canção exige, inclusive, a construção de parâmetros próprios de análise
e crítica.

Este lugar impreciso, entre a fala e o canto, faz aparecer variantes
melódicas que servem como uma forma possível de tipificar a canção, com
especial atenção para duas delas: a passional e a temática, que
trataremos mais adiante, além de uma terceira, a figurativa ou
enunciativa. Tão importante são essas variantes que podemos, através
delas, categorizar uma série de canções, mesmo movimentos artísticos
ligados à canção popular, até a obra de artistas.

Mas o que é, afinal de contas, esse limiar impreciso de que falamos, que
permite gerar a linguagem da canção, a própria figura do cancionista?

É a entoação. A entoação é a entidade que permite a construção de uma
série de relações entre a fala e o canto e, com isso, induz à criação
das melodias que geram a forma canção. Vejam que existe a primazia da
palavra, não da palavra isolada, mas entoada, da palavra que gera
variações melódicas. Essa descoberta, central para a conceituação de
Tatit, se deu através de um insight, ao ver uma interpretação de
Gilberto Gil a partir de uma canção interpretada por Germano Mathias,
como se nota na seguinte passagem de ``O século xx em foco''

\emph{Meu espanto não decorria do fato de essa canção exibir, de ponta a
ponta, seu vínculo com a fala, mas da hipótese, então bem nebulosa, de
outras canções, totalmente distintas, como Travessia, Garota de Ipanema
ou Quero Que Tudo Vá pro Inferno, camuflarem esse mesmo vínculo. De
qualquer forma, o centro do problema deslocava-se para fora da música e
da poesia, embora ambas participassem das etapas de criação. Passei a
enxergar a canção como produto de uma dicção. E mais que pela fala
explícita, passei a me interessar pela fala camuflada em tensões
melódicas.}

A entoação gera duas formas principais da canção: a forma passionalizada
e a forma tematizada, além de uma terceira, a figurativa. A primeira se
caracteriza por uma extensão da duração e da frequência das frases
musicais, com uma primazia das vogais alongadas e expandidas, a denotar
uma dimensão mais própria do Ser, no sentido da paixão, mais da
passionalidade do que da ação. A segunda, por sua vez, se caracteriza
por uma redução da duração e da frequência das frases musicais, com a
primazia se associando ao jogo de consoantes que vão criando quebras
rítmicas, que também podem ser nomeadas como temas, daí o termo
"tematizada". Com isso, coloca em primeiro plano a ação. Já a terceira,
por sua vez, enfatiza a figuração de personagens e tipificações, com as
canções vinculadas ao ``aqui e agora'', como se o artista estivesse
falando diretamente a um interlocutor, comentando uma situação
específica, numa conversa direta. Podemos pensar aqui em canções como
``Acertei no milhar'', ``Conversa de Botequim'', ou ``Amigo é pra essas
coisas''

No caso das duas anteriores, entre a paixão e a ação, a extensão vogal e
a contenção das consoantes, a expansão da duração e freqüência e a sua
redução e concentração, o nível de operacionalização parece ser maior.
Podemos ver isso, por exemplo, em muitos exemplos de canções e, mesmo,
de estilos e subgêneros. As serestas e tangos cantados por vozes como as
de Vicente Celestino e as marchinhas de carnaval em vozes como as de
Almirante; O samba canção e o samba breque; Francisco Alves em
contraponto a Mário Reis; Araci de Almeida em contraponto à dicção mais
solta e tematizada de Carmem Miranda; A diferença de andamento entre
Disparada (Geraldo Vandré/Theo de Barros) e A banda (Chico Buarque),
para usar exemplos dados pelo próprio autor.

É ele mesmo que o diz, aliás, que a descoberta do princípio da entoação,
chamemos assim, fez com que pudesse vislumbrar um método próprio de
análise da canção que fosse válido para canções tão díspares como
``Travessia'' (Milton Nascimento/Fernando Brant), ``Garota de Ipanema''
(Tom Jobim/Vinícius de Moraes) ou ``Quero que tudo vá pro inferno''
(Eramos Carlos/Roberto Carlos), como mencionamos no trecho citado mais
acima.

O mesmo pode se notar, o que é bastante interessante, no que diz
respeito aos movimentos artísticos ligados à canção popular. Dois deles
em especial podem ser considerados como as balizas do século XX na
canção brasileira, segundo a conceituação de Tatit. De um lado, a Bossa
Nova, como exemplar maior da variante temática, \emph{locus} do
exercício de lapidação e de apagamento das arestas exageradas e
rebarbativas da variante passional. O momento de concentração é também o
momento de seleção e mediação crítica mais acentuada e, digamos assim,
rigorosa.

No entanto, levado ao extremo, a variante temática acaba por excluir
expressões de muita vitalidade e força, especialmente aquelas associadas
à variante passional. Assim, com o tropicalismo temos uma nova abertura,
um novo esgarçamento da forma, com a inclusão e a explicitação da
pluralidade de formas de se fazer canção, trazendo novamente ao centro a
variante passionalizada, muito comum na nobre linhagem da canção
romântica brasileira, por exemplo.

Aqui se situam os aspectos formais da linguagem da canção. Estes
aspectos se misturam a outros, como as dimensões históricas e técnicas,
não menos fundamentais. A canção, tal qual a conhecemos, é um produto do
século XX, tem relação direta com algumas mudanças tecnológicas no modo
de gravação e reprodução da música, diretamente vinculados à indústria
cultural moderna.

Ela se desenvolve, no caso do Brasil, entre o lundu, associado ao
conjunto de sonoridades dos batuques dos escravizados neoafricanos e a
modinha, como forma musical vinda da tradição européia. Temos, assim,
uma aproximação que gerou não uma síntese, necessariamente, mas uma
outra forma artística. Tanto na sua dimensão harmônico/melódica, quanto
na dimensão da poética. Uma poética da canção que vai se construindo não
como poesia letrada "respeitável" dos salões das modinhas, muito menos
como variações eivadas de comicidade e malícia dos lundus, mas como
outra coisa, que unisse ambas as possibilidades, mas sempre tendo a
entoação como primado.

Mas este processo histórico, que remonta aos fins de século XIX, e as
primeiras décadas do século XX, têm uma dimensão tecnológica
fundamental. É que o período coincide com a criação de formas de
gravação e registro técnico das canções, e com a chegada dos aparelhos
fonomecânicos ao Brasil. Assim, no momento em que está em gestão uma
nova forma de entoar, nem diretamente vinculada ao lundu, nem
diretamente vinculada às modinhas, surge no país uma forma técnica de
registro musical que vai servir de modo perfeito. O registro fonográfico
nos discos passa a ter um papel equivalente ao cancionista à partitura
para o músico erudito.

Em suma, existe um vínculo orgânico entre o surgimento da canção moderna
e o desenvolvimento das técnicas de gravação e registro fonográficos,
ambos consolidados no Brasil a partir da década de 30, como podemos ver
claramente nessa passagem do mesmo texto

\emph{Dos batuques emanavam um volume de som incompatível com os parcos
recursos de gravação implantados pelos primeiros grupos de fonógrafos
que aportaram no Rio de Janeiro. De outra parte, os mestres do chorinho
e de outros gêneros de música escrita não viam razão para trocar sua
forma precisa de registro em partitura pelos meios fonomecânicos
rudimentares que jamais expressariam todos os matizes musicais de suas
composições. As canções, ao contrário, por estarem baseadas numa
oralidade de natureza instável (também já vimos que a entoação da fala
tende a desaparecer assim que a mensagem do texto é transmitida),
precisavam da gravação como recurso de fixação das obras que, até então,
quando não se perdiam nas rodas de brincadeira, passavam a depender
exclusivamente da boa memória de seus praticantes.}

\textbf{Os estudos de caso: Tom Jobim e Itamar Assumpção}

É a partir desse modelo, das variantes tipológicas da canção, em seus
aspectos formais e históricos, que veremos a análise de dois estudos de
caso expressivos. O primeiro se debruça sobre a dicção do maestro
soberano Tom Jobim. Músico com formação erudita, ao mesmo tempo em que
excelente cancionista. Tom Jobim foi um dos mais importantes
modernizadores da canção brasileira e, também, criou peças sinfônicas. O
segundo, Itamar Assumpção, um dos artistas mais inventivos da
\emph{Vanguarda Paulista}, das movimentações centrais para a renovação
formal da música e canção brasileira.

É muito importante explicitar a diferença significativa entre os dois
artistas da canção. Tom Jobim é um dos raríssimos músicos com formação
erudita e que é também exímio cancionista. Ou melhor, que colocou a
música a serviço do exercício cancional. Itamar Assumpção é o
cancionista por excelência, suas canções explicitam o papel da entoação
como poucos entre seus pares e faz dela motivo para canções pop e
experimentais, ou mesmo inclassificáveis, podendo ser considerado um dos
maiores malabaristas da palavra entoada.

No texto sobre Tom Jobim, vale ver com vagar as análises finas de
canções como ``Luíza'' e ``Corcovado'', além do trabalho de atenção
histórica ao seu lugar no processo de modernização da canção brasileira
atenta aos movimentos da canção americana do período. A dicção de Tom
Jobim assim incorpora aspectos modernizantes da canção americana da
primeira metade do século XX, insere no núcleo de feitura da canção
brasileira e gera, com isso, variantes harmônicas na tessitura melódica
das canções. A inserção dessas variantes responde, é preciso se dizer,
às necessidades da canção e não propriamente da música. Temos aqui,
novamente, a presença da entoação como princípio de estruturação da
linguagem da canção. E temos, a partir daí, variações harmônicas
complexas e estruturações melódicas não menos complexas.

Mas sobre a dicção de Tom Jobim há muito mais a dizer. Também vale ver
"Garota de Ipanema", um belo exemplo dos usos das variantes temáticas e
passionalizadas internas à própria canção. Assim, na primeira parte,
temos um exemplo claro da variante temática, quando parece mesmo que o
jogo rítmico, melódico e silábico acompanha o passo da garota tipificada
na canção, como se o mimetizasse. Há, inclusive, uma sensação de alegria
e leveza, formalmente bem resolvidas, e sem drama ou paixão exasperada.

Já na segunda parte a situação se modifica. Temos a passagem da variante
temática para a variante passionalizada. A relação entre entoação e
melodia se modifica. Agora vemos as vogais se alongarem fazendo com que
a própria melodia se expanda e colocando no centro de tudo a paixão e a
tristeza. O sujeito da canção se pergunta: ``Ah, por que estou tão
sozinho?''/Ah, por que tudo é tão triste?" para voltar, depois, para a
variante temática e assim finalizar a canção.

A análise de Itamar Assumpção faz uma gênese de toda a sua carreira, dos
primeiros discos, Beleléu, Leléu, Eu (1980), Às próprias custas S.A
(1983) e Sampa Midinight (1986), passando por ``Intercontinental Quem
diria! Era só o que faltava!!! (1988)'', Bicho de 7 cabeças I, II e III
(1993) e Pretobrás (1998). O que se nota é uma verdadeira ``Transmutação
do artista'', daí o título do texto. Transmutação que se dá entre o
primeiro momento de suas composições, com o vínculo direto entre as
canções e a própria figura, cênica e corporal, de Itamar Assumpção. As
composições possuem uma relação de organicidade com o personagem
construído pelo autor, nomeado de diferentes modos, mas sempre como a
figura do anti-herói, no ciclo ``Nego Dito'', que compõe os primeiros
discos. A partir do ``Intercontinental! Quem diria! Era só isso o que
faltava!'', começa a se notar uma mudança gradativa. O personagem vai
perdendo espaço, e a motivação das canções começam a se vincular a elas
mesmas, às suas próprias dinâmicas e tramas formais. É como se estivesse
havendo uma espécie de deslocamento de sentido, implicando numa
multiplicidade possível de personagens que poderiam ser ancorados nas
composições.

A própria obra se abre para a possibilidade de outros intérpretes, algo
quase impossível no período do ``projeto Beleléu'', tamanha força de
sintonia entre o autor e sua obra, tamanha necessidade de boa realização
das composições com a experimentação cênica do próprio Itamar. Essa
mudança se consolida nos álbuns posteriores, ``Bicho de sete cabeças'' e
``Pretobrás'', nas participações de artistas como Rita Lee, Luiz
Melodia, Jards Macalé, na profusão de figurações sociais em Pretobrás.

\textbf{Os ensaios e a voz sem coral}

Por fim, o livro finaliza com o terceiro bloco, com três textos que
mantém o tema da canção, do cancionista, da singularidade e autonomia
dessa linguagem artística. No primeiro, ``O momento da criação em canção
popular'' (1994), vale notar a questão da criação na canção popular,
através do exemplo da forma como Chico Buarque compôs, com Francis Hime,
o samba ``Vai passar'', que viria a se tornar um dos hinos da abertura
política do país. Mas que nasce como desdobramento formal das tramas
sonoras da melodia que vai, a seu modo, sugerindo sílabas, palavras e
versos.

Já o segundo texto, ``Vocação e perplexidade do cancionista'' (1983),
explicita a dificuldade de situar, e de se situar, que tem o cancionista
em relação ao seu lugar no campo artístico, intelectual e, mesmo,
social, devido à sua condição algo imprecisa, nem músico, nem literato.
E isso é válido também para o meio acadêmico. O cancionista pode ser
tudo, estudante de administração de empresas, filosofia, sociologia,
letras, biologia e, quem o sabe, até mesmo de música. Trata se de um
texto de 83, escrito para o caderno cultural da Folha de São Paulo, em
que já se nota em forma embrionária, e dentro dos limites de um texto
feito para o jornal, o que viria a ser a sua tese própria, com
conceituação mais precisa, sobre a singularidade da linguagem da canção,
tal qual vimos nos dois primeiros textos desse livro.

Por fim, fechamos o livro com ``Cancionistas invisíveis'' (2006), um
texto que concentra o tema nas questões do próprio mercado de canções,
ou das tramas complexas da recepção do público. Tatit destaca o lugar
dos cancionistas que não têm necessariamente um público de massa, que se
expressam em shows em pequenos teatros, mantendo, no entanto, um público
permanente, fiel e sempre atento.

Aqui o crítico, professor, artista da canção parece ecoar uma das suas
mais belas composições, ``Show'' (com Fábio Tagliaferri). Essa canção
foi apresentada, inicialmente, num festival da Globo, no ano 2000, com
interpretação da Ná Ozzetti. Também se tornou título de um álbum da
mesma cantora, com um repertório de canções da Época de Ouro, além dessa
canção.

Para nossos interesses aqui e, como forma de fechar essa pequena
apresentação, podemos dizer que o texto e a canção reafirmam, de todo
modo que, no fundo, com público de massa ou não, \emph{todos entoam}

E quem sonhou

Sofreu, chorou

Pode fazer

De uma só voz

Um show

Pode não ser

Um megashow

Um festival

Com multidões

Mas quem chorou

Já tem na voz

Um show
